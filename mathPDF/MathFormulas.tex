\documentclass[11pt]{article}
\renewcommand{\baselinestretch}{1.05}
\usepackage{amsmath,amsthm,verbatim,amssymb,amsfonts,amscd,graphicx,xcolor}
\usepackage{graphics,epstopdf,hyperref,mathtools,subcaption}

\topmargin0.0cm
\headheight0.0cm
\headsep0.0cm
\oddsidemargin0.0cm
\textheight23.0cm
\textwidth16.5cm
\footskip1.0cm
\setlength{\parindent}{0pt}
\setlength{\parskip}{1em}
\theoremstyle{plain}
\newtheorem{theorem}{Theorem}
\newtheorem{corollary}{Corollary}
\newtheorem{lemma}{Lemma}
\newtheorem{proposition}{Proposition}
\newtheorem*{surfacecor}{Corollary 1}
\newtheorem{conjecture}{Conjecture} 
\newtheorem{question}{Question} 
\theoremstyle{definition}
\newtheorem{definition}{Definition}
\DeclareMathOperator{\Tr}{Tr}
\newcommand{\rank}{\operatorname{rank}}

 \begin{document}

\title{Towards discovering structure in gene expression across properties\\
\small MDI-Barn Raising}
\author{Lisa, Harriet, Dave, Yuan, Oliver}
\maketitle

\section{Simulate Properties across Species}
Assume for each property $P_i$, the species can be classified into $n$ groups with distribution in a specific range in $[0, 1]$.



\end{document}